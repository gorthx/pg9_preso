\documentclass{beamer}

\definecolor{pgdarkblue}{HTML}{0064a5}
\definecolor{pgbaseblue}{HTML}{336791}
\definecolor{pglightblue}{HTML}{008bb9}

\definecolor{pgdarkgrey}{HTML}{848484}
\definecolor{pgbasegrey}{HTML}{666666}
\definecolor{pglightgrey}{HTML}{585858}

\definecolor{pgdarkorange}{HTML}{cc3b03}
\definecolor{pgbaseorange}{HTML}{d24b03}
% Don't know why the HTML code causes latex to error.
%\definecolor{pglightorange}{HTML}{f26522}
\definecolor{pglightorange}{RGB}{242,101,34}

\setbeamercolor{item}{fg=white,bg=pgdarkblue}
\setbeamercolor{normal text}{fg=white,bg=pgdarkblue}
\setbeamercolor{titlelike}{fg=white,bg=pgdarkblue}
\usepackage{ulem}

% Title page starts here.
\title{PostgreSQL 9.0}
\subtitle{KEWL STUFF U CAN DO}
\author{Mark Wong - markwkm@gmail.com\\Gabrielle Roth - gorthx@gmail.com}
\institute{PDXPUG}
\date{June 17, 2010}

\begin{document}

\frame{\titlepage}

\frame[containsverbatim]
{
  \frametitle{Introductions}

  \begin{verbatim}
          __     __
         /  \~~~/  \  . o O ( PostgreSQL? )
   ,----(     oo    )
  /      \__     __/
 /|         (\  |(
^ \   /___\  /\ |
   |__|   |__|-"
  \end{verbatim}
}
\frame
{
  \frametitle{PostgreSQL}

  \begin{itemize}
  \item[-] Current version 8.4.4
  \item[-] 9.0 is in beta2
  \item[-] Get it here:\\
  http://www.postgresql.org/developer/beta\\
  http://www.postgresql.org/ftp/source/v9.0beta2/
  \end{itemize}
}

\frame
{
  \frametitle{contents}

  full list of features:\\
  http://developer.postgresql.org/pgdocs/postgres/\\
  release-9-0.html
}

\frame
{
  \begin{center}
  \huge{Gab's Top}
  \sout{10}
  \huge{11}
  \end{center}
}

\frame
{
  \frametitle{Install server-side language PL/pgSQL by default (Bruce Momjian)}
  \begin{center}
  Yes.
  \end{center}
}

\frame
{
  \frametitle{pg\_table\_size and pg\_indexes\_size (Bernd Helmle)}
  \begin{center}
  Discuss.
  \end{center}
}

\frame
{
  \frametitle{vacuumdb --analyze-only (Bruce Momjian)}
  \begin{center}
  Hint:  You won't get any output.
  \end{center}
}

\frame
{
  \frametitle{Allow function calls to supply parameter names and match them to named parameters in the function definition (Pavel Stehule)}

(Demo.)
}

\frame
{
  \frametitle{/contrib/passwordcheck (Laurenz Albe)}
  \begin{center}
  Discuss.
  \end{center}
}

\frame[containsverbatim]
{
  \frametitle{RADIUS Authentication (Magnus Hagander)}
  \begin{verbatim}
#host  database  user  CIDR-address  auth-method  [auth-options]
host   all       all   0.0.0.0/0     radius       \
radiusserver=1.2.3.4 radiussecret=canttouchthis \
radiusport=1234  radiusidentifier=gabscooldatabase
  \end{verbatim}
}

\frame[containsverbatim]
{
  \frametitle{Add samehost and samenet designations to pg\_hba.conf (Stef Walter)}
  \begin{verbatim}
# IPv4 local connections:
host all  all             127.0.0.1/32            md5
host all  all             samehost                md5
host all  all             samenet                 md5
  \end{verbatim}
}

\frame[containsverbatim]
{
  \frametitle{Log changed parameter values when postgresql.conf is reloaded (Peter Eisentraut)}
  \begin{verbatim}
LOG:  parameter "log_rotation_age" changed to "7d"
LOG:  parameter "port" cannot be changed without restarting the server
LOG:  parameter "log_rotation_age" changed to "1d"
LOG:  parameter "log_min_messages" changed to "notice"
LOG:  parameter "autovacuum" changed to "off"
<2010-04-22 14:00:50 PDT  >LOG:  parameter "log_line_prefix" changed to "<%t %d %u>"
<2010-04-22 14:00:50 PDT  >LOG:  parameter "log_statement" changed to "mod"
  \end{verbatim}
}

\frame[containsverbatim]
{
  \frametitle{Add an SQL state option (\%e) to log\_line\_prefix (Guillaume Smet)}
  e.g.:
  \begin{verbatim}
log_line_prefix = '<%t %d %u %e>'
  \end{verbatim}
  gives you something like:
  \begin{verbatim}
<2010-04-22 21:02:02 PDT postgres postgres 00000> \
 LOG:  statement: ALTER TABLE drop_column_demo DROP COLUMN IF EXISTS b;
<2010-04-22 21:02:54 PDT postgres postgres 00000> \
 LOG:  statement: GRANT SELECT on drop_column_demo to groth;
<2010-04-22 21:02:54 PDT postgres postgres 42704> \
 ERROR:  role "groth" does not exist
  \end{verbatim}
}

\frame[containsverbatim]
{
  \frametitle{Allow EXPLAIN output in XML, JSON, and YAML formats
              (Robert Haas, Greg Sabino Mullane)}

  \begin{verbatim}
EXPLAIN 
SELECT * FROM user_history ORDER BY "timestamp";
QUERY PLAN                                 
---------------------------------------------------------------------------
 Sort  (cost=18494.32..18852.33 rows=143204 width=34)
   Sort Key: "timestamp"
   ->  Seq Scan on user_history  (cost=0.00..2559.04 rows=143204 width=34)
(3 rows)
  \end{verbatim}
}

\frame[containsverbatim]
{
  \frametitle{Con't}

  \begin{verbatim}
EXPLAIN (FORMAT XML)
SELECT * FROM user_history ORDER BY "timestamp";
QUERY PLAN                         
------------------------------------------------------------
<explain xmlns="http://www.postgresql.org/2009/explain">  +
  <Query>                                                 +
    <Plan>                                                +
      <Node-Type>Sort</Node-Type>                         +
      <Startup-Cost>18494.32</Startup-Cost>               +
      <Total-Cost>18852.33</Total-Cost>                   +
      <Plan-Rows>143204</Plan-Rows>                       +
      <Plan-Width>34</Plan-Width>                         +
      <Sort-Key>                                          +
        <Item>"timestamp"</Item>                          +
      </Sort-Key>                                         +
...
  \end{verbatim}
}

\frame[containsverbatim]
{
  \frametitle{Con't}
  \begin{verbatim}
...
      <Plans>                                             +
        <Plan>                                            +
          <Node-Type>Seq Scan</Node-Type>                 +
          <Parent-Relationship>Outer</Parent-Relationship>+
          <Relation-Name>user_history</Relation-Name>     +
          <Alias>user_history</Alias>                     +
          <Startup-Cost>0.00</Startup-Cost>               +
          <Total-Cost>2559.04</Total-Cost>                +
          <Plan-Rows>143204</Plan-Rows>                   +
          <Plan-Width>34</Plan-Width>                     +
        </Plan>                                           +
      </Plans>                                            +
    </Plan>                                               +
  </Query>                                                +
</explain>
(1 row)
  \end{verbatim}
}

\frame[containsverbatim]
{
  \frametitle{Con't}
  \begin{verbatim}
EXPLAIN (FORMAT YAML)
SELECT * FROM user_history ORDER BY "timestamp";
QUERY PLAN              
-------------------------------------
 - Plan:                            +
     Node Type: Sort                +
     Startup Cost: 18494.32         +
     Total Cost: 18852.33           +
     Plan Rows: 143204              +
     Plan Width: 34                 +
     Sort Key:                      +
       - "\"timestamp\""            +
...
  \end{verbatim}
}

\frame[containsverbatim]
{
  \frametitle{Con't}
  \begin{verbatim}
     Plans:                         +
       - Node Type: Seq Scan        +
         Parent Relationship: Outer +
         Relation Name: user_history+
         Alias: user_history        +
         Startup Cost: 0.00         +
         Total Cost: 2559.04        +
         Plan Rows: 143204          +
         Plan Width: 34
(1 row)
  \end{verbatim}
}

\frame[containsverbatim]
{
  \frametitle{Add support for to\_char() scientific notation output ('EEEE') (Pavel Stehule, Brendan Jurd)}
  \begin{verbatim}
postgres=# create table exponents_demo
(example numeric);
CREATE TABLE
postgres=# INSERT INTO exponents_demo
VALUES
  ('10000000'),
  ('2.54'),
  ('3.1415'),
  ('666'),
  ('6.02e+23'),
  ('5.29e-11'),
  ('8675309')
;
  \end{verbatim}
}

\frame[containsverbatim]
{
  \frametitle{Con't}
  \begin{verbatim}
postgres=# SELECT * from exponents_demo ;
       example 
--------------------------
                 10000000
                     2.54
                   3.1415
                      666
 602000000000000000000000
          0.0000000000529
                  8675309
(7 rows)
  \end{verbatim}
}

\frame[containsverbatim]
{
  \frametitle{Con't}
  \begin{verbatim}
postgres=# SELECT to_char(example, '9.9EEEE')
AS lookyhere
FROM exponents_demo ;
 lookyhere
-----------
  1.0e+07
  2.5e+00
  3.1e+00
  6.7e+02
  6.0e+23
  5.3e-11
  8.7e+06
(7 rows)
  \end{verbatim}
}

\frame
{
  \begin{center}
  \huge{Selena's picks}
  \end{center}
}

\frame[containsverbatim]
{
  \frametitle{Speed up CREATE DATABASE by deferring flushes to disk (Andres Freund, Greg Stark)}
(Demo.)
}

\frame[containsverbatim]
{
  \frametitle{Add WHERE clause to CREATE TRIGGER to allow control over
              whether a trigger is fired (Takahiro Itagaki)}
psst\ldots It's actually a "When" clause\ldots
  \begin{verbatim}
CREATE TRIGGER name { BEFORE | AFTER } { event [ OR ... ] }
    ON table [ FOR [ EACH ] { ROW | STATEMENT } ]
    [ WHEN ( condition ) ]
    EXECUTE PROCEDURE function_name ( arguments )
  \end{verbatim}
}

\frame[containsverbatim]
{
  \frametitle{Allow NOTIFY to pass an optional string to listeners
              (Joachim Wieland)}

  \begin{verbatim}
lfnw-9.0=# LISTEN l9_0;
LISTEN
lfnw-9.0=# NOTIFY l9_0, 'need coffee...';
NOTIFY
Asynchronous notification "l9_0"
with payload "need coffee..."
received from server process with PID 10480.
  \end{verbatim}
}

\frame[containsverbatim]
{
  \frametitle{Have columns defined with storage type MAIN 
  remain on the main heap page unless it cannot fit (Kevin Grittner)}
  \begin{verbatim}
8k page size
Before: a tuple might get TOAST'd prematurely
LIVE DEMO!
  \end{verbatim}
}

\frame
{
  \frametitle{Add pg\_stat\_reset\_single\_table\_counters() 
  and pg\_stat\_reset\_single\_function\_counters() 
  to allow the resetting of statistics counters for 
  individual tables and indexes (Magnus Hagander)}

  (Discuss.)
}

\frame
{
  \frametitle{Add pg\_last\_xlog\_receive\_location() 
  and pg\_last\_xlog\_replay\_location(), which can be used to monitor 
  standby server WAL activity (Simon, Fujii Masao, Heikki)}

  (Discuss.)
}

\frame
{
  \frametitle{Hot Standby/Streaming Replication}

  \ldots drumroll \ldots
}

\frame
{
  \frametitle{And now for something completely different...}

  \begin{center}
  Pg Android App Contest!
  \end{center}
  \url{http://wiki.postgresql.org/wiki/AndroidAppContest}
}

\frame
{
  \frametitle{Odds \& Ends}
  \begin{itemize}
  \item[] GRANT/REVOKE IN SCHEMA (Petr Jelinek)
  \item[] Add deferrable unique constraints (Dean Rasheed)
  \item[] Remove the use of flat files for system table bootstrapping (Tom, Alvaro)
  \item[] DROP COLUMN IF EXISTS (Andreas Freund)
  \item[] 64-bit Windows support (Tsutomu Yamada, Magnus Hagander)
  \item[] Improvements to PL/Perl (Tim Bunce)
  \item[] Prevent overwriting of psql's command-line history if two psql sessions are run simultaneously (Tom Lane)
  \item[] Anonymous functions with 'DO' (Petr Jelinek, Joshua Tolley, Hannu Valtonen)
  \end{itemize}
}

\frame
{
  \frametitle{Final Notes}

  These slides:\\
  \begin{itemize}
  \item[] \url{http://www.slideshare.net/markwkm}
  \item[] \url{http://github.com/gorthx/pg9\_preso}
  \end{itemize}

  Try it!\\
  \begin{itemize}
  \item[] \url{http://www.postgresql.org/developer/beta}
  \item[] \url{http://www.postgresql.org/ftp/source/v9.0beta2/}
  \item[] \url{http://wiki.postgresql.org/wiki/HowToBetaTest}
  \end{itemize}
}

\frame[containsverbatim]
{
  \frametitle{Acknowledgements}

  Post-it Extra-Sticky Notes\\
  Hayley Jane Wakenshaw\\
  \begin{verbatim}
          __     __
         /  \~~~/  \  . o O ( Thank you! )
   ,----(     oo    )
  /      \__     __/
 /|         (\  |(
^ \   /___\  /\ |
   |__|   |__|-"
  \end{verbatim}

}

\frame
{
  \frametitle{License}

  This work is licensed under a Creative Commons Attribution 3.0
  Unported License. To view a copy of this license, (a) visit
  \url{http://creativecommons.org/licenses/by/3.0/us/}; or, (b) send a
  letter to Creative Commons, 171 2nd Street, Suite 300, San Francisco,
  California, 94105, USA.
}

\end{document}
